\documentclass[a4paper,10pt,landscape]{article}
\usepackage{multicol}
\usepackage{calc}
\usepackage{ifthen}
\usepackage[landscape]{geometry}


% Times for rm and math | Helvetica for ss | Courier for tt
\usepackage{mathptmx} % rm & math
\usepackage[scaled=0.90]{helvet} % ss
\usepackage{courier} % tt
\normalfont
\usepackage[T1]{fontenc}


%% Palatino for rm and math | Helvetica for ss | Courier for tt
%\usepackage{mathpazo} % math & rm
%\linespread{1.05}        % Palatino needs more leading (space between lines)
%\usepackage[scaled]{helvet} % ss
%\usepackage{courier} % tt
%\normalfont
%\usepackage[T1]{fontenc}

%% Euler for math | Palatino for rm | Helvetica for ss | Courier for tt
%\renewcommand{\rmdefault}{ppl} % rm
%\linespread{1.05}        % Palatino needs more leading
%\usepackage[scaled]{helvet} % ss
%\usepackage{courier} % tt
%\usepackage{euler} % math
%%\usepackage{eulervm} % a better implementation of the euler package (not in gwTeX)
%\normalfont
%\usepackage[T1]{fontenc}



% To make this come out properly in landscape mode, do one of the following
% 1.
%  pdflatex latexsheet.tex
%
% 2.
%  latex latexsheet.tex
%  dvips -P pdf  -t landscape latexsheet.dvi
%  ps2pdf latexsheet.ps


% If you're reading this, be prepared for confusion.  Making this was
% a learning experience for me, and it shows.  Much of the placement
% was hacked in; if you make it better, let me know...


% 2008-04
% Changed page margin code to use the geometry package. Also added code for
% conditional page margins, depending on paper size. Thanks to Uwe Ziegenhagen
% for the suggestions.

% 2006-08
% Made changes based on suggestions from Gene Cooperman. <gene at ccs.neu.edu>


% To Do:
% \listoffigures \listoftables
% \setcounter{secnumdepth}{0}


% This sets page margins to .5 inch if using letter paper, and to 1cm
% if using A4 paper. (This probably isn't strictly necessary.)
% If using another size paper, use default 1cm margins.
\ifthenelse{\lengthtest { \paperwidth = 11in}}
	{ \geometry{top=.5in,left=.5in,right=.5in,bottom=.5in} }
	{\ifthenelse{ \lengthtest{ \paperwidth = 297mm}}
		{\geometry{top=1cm,left=1cm,right=1cm,bottom=1cm} }
		{\geometry{top=1cm,left=1cm,right=1cm,bottom=1cm} }
	}

% Turn off header and footer
\pagestyle{empty}
 

% Redefine section commands to use less space
\makeatletter
\renewcommand{\section}{\@startsection{section}{1}{0mm}%
                                {-1ex plus -.5ex minus -.2ex}%
                                {0.5ex plus .2ex}%x
                                {\normalfont\large\bfseries}}
\renewcommand{\subsection}{\@startsection{subsection}{2}{0mm}%
                                {-1explus -.5ex minus -.2ex}%
                                {0.5ex plus .2ex}%
                                {\normalfont\normalsize\bfseries}}
\renewcommand{\subsubsection}{\@startsection{subsubsection}{3}{0mm}%
                                {-1ex plus -.5ex minus -.2ex}%
                                {1ex plus .2ex}%
                                {\normalfont\small\bfseries}}
\makeatother

% Define BibTeX command
\def\BibTeX{{\rm B\kern-.05em{\sc i\kern-.025em b}\kern-.08em
    T\kern-.1667em\lower.7ex\hbox{E}\kern-.125emX}}

% Don't print section numbers
\setcounter{secnumdepth}{0}


\setlength{\parindent}{0pt}
\setlength{\parskip}{0pt plus 0.5ex}


% -----------------------------------------------------------------------

\begin{document}

\raggedright
\footnotesize
\begin{multicols}{3}


% multicol parameters
% These lengths are set only within the two main columns
%\setlength{\columnseprule}{0.25pt}
\setlength{\premulticols}{1pt}
\setlength{\postmulticols}{1pt}
\setlength{\multicolsep}{1pt}
\setlength{\columnsep}{2pt}

\begin{center}
     \Large{\textbf{VampirTrace Cheat Sheet}} \\
\end{center}

\section{Commandline arguments}
\begin{tabular}{@{}ll@{}}
\texttt{-vt:help}            & Show the help message.\\
\texttt{-vt:cc  <cmd>}       & Set the underlying compiler command.\\
\texttt{-vt:inst <type>}     & Set the instrumentation type.\\
\hspace{5ex} compinst          & fully-automatic by compiler\\
\hspace{5ex} manual            & manual by using VampirTrace's API\\
\hspace{5ex} dyninst           & binary by using Dyninst (www.dyninst.org) \\
\texttt{-vt:opari <args>}    & Set options for OPARI command.\\
\texttt{-vt:<par-type>}      & Force application's parallelization type.\\
\hspace{5ex} seq               & sequential\\
\hspace{5ex} mpi               & parallel (uses MPI)\\
\hspace{5ex} mt                & arallel (uses OpenMP/POSIX threads)\\
\hspace{5ex} hyb               & hybrid parallel (MPI + Threads)\\
\texttt{-vt:verbose}         & Enable verbose mode.\\
\texttt{-vt:show}            & dry-run, shows call to compiler.
\end{tabular}

\section{Compiler Wrappers}
environment variablesaffecting the compiler wrappers:\\
\begin{tabular}{@{}ll@{}}
\texttt{VT\_INST}            & Instrumentation type (equivalent to \texttt{-vt:inst})\\
\texttt{VT\_CC}              & C compiler command (equivalent to \texttt{-vt:cc})\\
\texttt{VT\_CFLAGS}          & C compiler flags\\
\texttt{VT\_LDFLAGS}         & Linker flags\\
\texttt{VT\_LIBS}            & Libraries to pass to the linker
\end{tabular}

\subsection{Serial programs}
\begin{tabular}{@{}ll@{}}
original:              &  \tt gfortran hello.f90 -o hello  \\
vampir:  &  \tt \textbf{vtf90} hello.f90 -o hello  \\
\end{tabular}

\subsection{MPI parallel programs}
\begin{tabular}{@{}ll@{}}
original:              &  \tt mpicc hello.c -o hello  \\
vampir:  &  \tt \textbf{vtcc -vt:cc mpicc} hello.c -o hello  \\
\end{tabular}

\subsection{Threaded parallel programs}
\begin{tabular}{@{}ll@{}}
original:              &  \tt ifort <-openmp|-pthread> hello.f90 -o hi \\
vampir:  &  \tt \textbf{vtf90} <-openmp|-pthread> hello.f90 -o hi \\
\end{tabular}

\subsection{Hybrid MPI/Threaded parallel programs}
\begin{tabular}{@{}ll@{}}
original:              &  \tt mpif90 <-openmp|-pthread> hello.F90 \\
                       &  \tt -o hello \\
vampir:  &  \tt \textbf{vtf90 -vt:f90 mpif90} <-openmp|-pthread> \\
                       &  \tt hello.F90 -o hello\\
\end{tabular}

\subsection{Unification of Local Traces}
Set \texttt{OMP\_NUM\_THREADS} if needed\\
Threaded: \verb!vtunify <prefix>!\\
MPI: \verb!mpirun -np <nranks> vtunify-mpi <prefix>!

\section{Instrumentation}
\subsection{Manual Instrumentation}

\begin{multicols}{2}
C:\\
\verb!#include "vt_user.h"!\\
\verb!VT_USER_START("name");!\\
\verb!...!\\
\verb!VT_USER_END("name");!\\

Fortran:\\
\verb!#include "vt_user.inc"!\\
\verb!VT_USER_START('name')!\\
\verb!...!\\
\verb!VT_USER_END('name')!
\end{multicols}
\vspace{2ex}
Combined with automatic compiler instrumentation:\\
\texttt{vtcc \textbf{-DVTRACE} hello.c -o hello}\\[1ex]
Without compiler instrumentation:\\
\texttt{vtcc -vt:inst manual \textbf{-DVTRACE} hello.c -o hello}

\subsection{Tracing Calls to 3rd-Party Libraries}
Create a instrumented version of shared library:\\
\verb!vtlibwrapgen -g SDL -o wrap.c /.../SDL/*.h!\\
\verb!vtlibwrapgen --build --shared -o libwrap wrap.c!\\
\verb!LD_PRELOAD=$PWD/libwrap.so <executable>!

\subsection{Switching tracing on/off}
\texttt{VT\_ON}/ \texttt{VT\_OFF} start and stop the recording of events\\
check whether tracing is enabled: \texttt{VT\_IS\_ON}

\subsection{Trace buffer rewind}
Decide dynamically \textit{after} a section wether to record events
\verb!do step=1,number_of_time_steps!\\
\verb!    VT_SET_REWIND_MARK()!\\
\verb!    call compute_time_step(step)!\\
\verb!    if(finished_as_expected) VT_REWIND()!\\
\verb!end do!

\subsection{Intermediate buffer flush}
Force flushing of buffers: \texttt{VT\_BUFFER\_FLUSH}

\subsection{Intermediate time synchronisation}
Flush the buffer at \textit{!synchronized!} point:(eg. barrier): \texttt{VT\_TIMESYNC}
Don't forget to \texttt{export VT\_ETIMESYNC=yes}

\subsection{Intermediate counter update}
For counter values at any given point: \texttt{VT\_UPDATE\_COUNTER}

\subsection{User defined counter}
Allow recording of program variable values or any other numerical quantity.
It is identified by its name, the counter group, the type of its value (int or float) and the unit(e.g.~``GFlop/sec'').
\begin{multicols}{2}
C:\\
\verb!#include "vt_user.h"!\\
\verb!unsigned int id, gid;!\\
\verb!gid = VT_COUNT_GROUP_DEF!\\
\verb!                ("name");!\\
\verb!id = VT_COUNT_DEF("name",!\\
\verb!      "unit", type, gid);!\\

Fortran:\\
\verb!#include "vt_user.inc"!\\
\verb!integer :: id, gid!\\
\verb!VT_COUNT_GROUP_DEF!\\
\verb!     ('name', gid)!\\
\verb!VT_COUNT_DEF('name',!\\
\verb!'unit', type, gid, id)!!\\
\end{multicols}

All the uppercase entries start with \texttt{\textbf{VT\_COUNT\_}} that has been ommitted to fit on the cheatsheet.
\begin{tabular}{@{}l@{ }l@{ }l@{}}
\textbf{Fortran:} \\
\textbf{Type} & \textbf{Count call} & \textbf{Data type} \\ 
\texttt{TYPE\_INTEGER} &
	\texttt{INTEGER\_VAL} &
	integer (4 byte) \\
\texttt{TYPE\_INTEGER8} &
	\texttt{INTEGER8\_VAL} &
	integer (8 byte) \\
\texttt{TYPE\_REAL} &
	\texttt{REAL\_VAL} &
	real \\
\texttt{TYPE\_DOUBLE} &
	\texttt{DOUBLE\_VAL} &
	double precision \\

\textbf{C/C++:} \\
\textbf{Type} & \textbf{Count call} & \textbf{Data type} \\ 
\texttt{TYPE\_SIGNED} &
	\texttt{SIGNED\_VAL} &
	signed int (max. 64-bit) \\
\texttt{TYPE\_UNSIGNED} &
	\texttt{UNSIGNED\_VAL} &
	unsigned int (max. 64-bit) \\
\texttt{TYPE\_FLOAT} &
	\texttt{FLOAT\_VAL} &
	float \\
\texttt{TYPE\_DOUBLE} &
	\texttt{DOUBLE\_VAL} &
	double \\
\end{tabular}

\section{Environment Variables}
\begin{tabular}{@{}l@{}l@{}l@{}}
\texttt{ } & \textbf{Global Settings} & \texttt{ } \\
\texttt{VT\_APPPATH} & Path to the application executable. & -- \\
\texttt{VT\_BUFFER\_SIZE} & Internal event trace buffer. & 32M \\
\texttt{VT\_CLEAN} & Remove temporary trace files? & yes \\
\texttt{VT\_COMPRESSION} & Write compressed trace files?  & yes \\
\texttt{VT\_FILE\_PREFIX} & Prefix used for trace filenames. & \\
\texttt{VT\_FILE\_UNIQUE} & Unique naming? no, yes, [id] & no \\
\texttt{VT\_MAX\_FLUSHES} & Maximum number of buffer flushes. &	1 \\
\texttt{VT\_MAX\_THREADS} & Threads VT reserves resources for &	65536 \\
\texttt{VT\_PFORM\_GDIR} & Global directory to store final trace. & \texttt{./} \\
\texttt{VT\_PFORM\_LDIR} & Node-local temp directory & \texttt{/tmp/} \\
\texttt{VT\_UNIFY} & Unify local trace files afterwards? & yes \\
\texttt{VT\_VERBOSE} & Quiet (0), Critical (1), Information (2) & 1 \\
%\end{tabular}
%\begin{tabular}{@{}l@{}l@{}l@{}}
\texttt{ } & \textbf{Optional Features} & \texttt{ } \\
\texttt{VT\_CPUIDTRACE} & Tracing of core ID of a CPU? & no \\
\texttt{VT\_ETIMESYNC} & Enhanced timer synchronization? & no \\
\texttt{VT\_ETIMESYNC\_INTV} & Interval between two phases(seconds). & 120 \\
\texttt{VT\_IOLIB\_PATHNAME} & Library to use for LIBC I/O calls. & -- \\
\texttt{VT\_IOTRACE} & Tracing of application I/O calls? & no \\
\texttt{VT\_LIBCTRACE} & Tracing of fork/system/exec calls? & yes \\
\texttt{VT\_MEMTRACE} & Memory allocation counter? & no \\
\texttt{VT\_MODE} & TRACE, STAT, TRACE:STAT (both). & TRACE \\
\texttt{VT\_MPICHECK} & Correctness checking via UniMCI? & no \\
\texttt{VT\_MPICHECK\_ERREXIT} & Force trace write on error & no \\
\texttt{VT\_MPITRACE} &	Tracing of MPI events?  & yes \\
\texttt{VT\_OMPTRACE} & Tracing of OpenMP events & yes \\
\texttt{VT\_PTHREAD\_REUSE} & Reuse IDs of terminated Pthreads? & yes \\
\texttt{VT\_STAT\_INV} & Interval of profiling records & 0 \\
\texttt{VT\_STAT\_PROPS} & FUNC, MSG, COLLOP, ALL & ALL \\
\texttt{VT\_SYNC\_FLUSH} & Synchronized buffer flush? & no \\
\texttt{VT\_SYNC\_FLUSH\_LEVEL} & Minimum buffer fill level in percent. & 80 \\
\end{tabular}
\begin{tabular}{@{}l@{}l@{}l@{}}
\texttt{ } & \textbf{Counters} & \texttt{ } \\
\texttt{VT\_METRICS} & list of counter. & -- \\
\texttt{VT\_METRICS\_SEP} &	Separator string in VT\_METRICS. & : \\
\texttt{VT\_RUSAGE} & Colon-separated list of rusage counters & -- \\
\texttt{VT\_RUSAGE\_INTV} & Sample interval in ms. & 100 \\
%\end{tabular}
%\begin{tabular}{@{}l@{}l@{}l@{}}
\texttt{ } & \textbf{Filtering, Grouping} & \texttt{ } \\
\texttt{VT\_DYN\_BLACKLIST} & Blacklist file for Dyninst & -- \\
\texttt{VT\_DYN\_SHLIBS} & Colon-separated list of shared libraries & -- \\
\texttt{VT\_FILTER\_SPEC} & Name of function/region filter file. & -- \\
\texttt{VT\_GROUPS\_SPEC} & Name of function grouping file. & -- \\
\texttt{VT\_JAVA\_FILTER\_SPEC} & Name of Java specific filter file. & -- \\
\texttt{VT\_GROUP\_CLASSES} & Create Java class group automatically? & yes \\
\texttt{VT\_MAX\_STACK\_DEPTH} & Maximum stack level (0 = unlimited) & 0 \\
%\end{tabular}
%\begin{tabular}{@{}l@{}l@{}l@{}}
\texttt{ } & \textbf{Symbol List} & \texttt{ } \\
\texttt{VT\_GNU\_NM} & Command for symbols of object files & nm \\
\texttt{VT\_GNU\_NMFILE} & File with symbol list information. & -- \\
\end{tabular}

\section{Synchronized Buffer Flush}
Have Vampir flush buffers when 1 buffer reaches \texttt{VT\_SYNC\_FLUSH\_LEVEL} at the next MPI collective that is using \texttt{MPI\_COMM\_WORLD}.\\
\texttt{export VT\_SYNC\_FLUSH=yes}

\section{Enhanced Timer Synchronization}
This scheme inserts additional synchronization phases at appropriate
points in the program flow. Currently, VampirTrace makes use of all MPI collective
functions associated with \texttt{MPI\_COMM\_WORLD}. A LAPACK library with C wrapper support has to be provided.\\
\texttt{export VT\_ETIMESYNC=yes}\\
Additionally, set the interval in seconds between the synchronization phases.\\
\texttt{export VT\_ETIMESYNC\_INTV=600}\\
LAPACK libraries providing a C-LAPACK API that can be used by VampirTrace:\\
- CLAPACK (www.netlib.org/clapack)\\
- AMD ACML\\
- IBM ESSL\\
- Intel MKL\\
- SUN Performance Library\\
Note though that this is not needed for systems with a global timer.

\section{Hardware Performance Counters}
VampirTrace supports different hardware counter libraries and relies on those API's to gather the counter values. An exclamation mark ''!'' marks a counter to be measured as an absolute value.

\subsection*{PAPI Hardware Performance Counters}
\verb|VT_METRICS=PAPI_FP_OPS:PAPI_L2_TCM:!CPU_TEMP1|\\
\texttt{CPU\_TEMP1} is provided by the lm-sensors component. See \texttt{papi\_avail} and \texttt{papi\_native\_avail} for available counter.
\begin{scriptsize}
\begin{verbatim}
PAPI_L[1|2|3]_[D|I|T]C[M|H|A|R|W]    
              Level 1/2/3 data/instruction/total cache 
              misses/hits/accesses/reads/writes
PAPI_L[1|2|3]_[LD|ST]M    
              Level 1/2/3 load/store misses                       
PAPI_CA_SNP   Requests for a snoop                                
PAPI_CA_SHR   Req. for excl. access to shared cache line  
PAPI_CA_CLN   Req. for excl. access to clean cache line   
PAPI_CA_INV   Requests for cache line invalidation                
PAPI_CA_ITV   Requests for cache line intervention                
PAPI_BRU_IDL  Cycles branch units are idle                        
PAPI_FXU_IDL  Cycles integer units are idle                       
PAPI_FPU_IDL  Cycles floating point units are idle                
PAPI_LSU_IDL  Cycles load/store units are idle                    
PAPI_TLB_DM   Data translation lookaside buffer misses            
PAPI_TLB_IM   Instruction transl. lookaside buffer misses     
PAPI_TLB_TL   Total translation lookaside buffer misses           
PAPI_BTAC_M   Branch target address cache misses                  
PAPI_PRF_DM   Data prefetch cache misses                          
PAPI_TLB_SD   Translation lookaside buffer shootdowns             
PAPI_CSR_FAL  Failed store conditional instructions               
PAPI_CSR_SUC  Successful store conditional instructions           
PAPI_CSR_TOT  Total store conditional instructions                
PAPI_MEM_SCY  Cycles Stalled Waiting for memory accesses          
PAPI_MEM_RCY  Cycles Stalled Waiting for memory Reads             
PAPI_MEM_WCY  Cycles Stalled Waiting for memory writes            
PAPI_STL_ICY  Cycles with no instruction issue                    
PAPI_FUL_ICY  Cycles with maximum instruction issue               
PAPI_STL_CCY  Cycles with no instructions completed               
PAPI_FUL_CCY  Cycles with maximum instructions completed          
PAPI_BR_UCN   Unconditional branch instructions                   
PAPI_BR_CN    Conditional branch instructions                     
PAPI_BR_TKN   Conditional branch instructions taken               
PAPI_BR_NTK   Conditional branch instructions not taken           
PAPI_BR_MSP   Conditional branch inst. mispredicted        
PAPI_BR_PRC   Cond. branch inst. correctly predicted 
PAPI_FMA_INS  FMA instructions completed                          
PAPI_TOT_IIS  Instructions issued                                 
PAPI_TOT_INS  Instructions completed                              
PAPI_INT_INS  Integer instructions                                
PAPI_FP_INS   Floating point instructions                         
PAPI_LD_INS   Load instructions                                   
PAPI_SR_INS   Store instructions                                  
PAPI_BR_INS   Branch instructions                                 
PAPI_VEC_INS  Vector/SIMD instructions                            
PAPI_LST_INS  Load/store instructions completed                   
PAPI_SYC_INS  Synchronization instructions completed              
PAPI_FML_INS  Floating point multiply instructions                
PAPI_FAD_INS  Floating point add instructions                     
PAPI_FDV_INS  Floating point divide instructions                  
PAPI_FSQ_INS  Floating point square root instructions             
PAPI_FNV_INS  Floating point inverse instructions                 
PAPI_RES_STL  Cycles stalled on any resource    
PAPI_FP_STAL  Cycles the FP unit(s) are stalled 
PAPI_FP_OPS   Floating point operations         
PAPI_TOT_CYC  Total cycles                      
PAPI_HW_INT   Hardware interrupts               
\end{verbatim} 
\end{scriptsize}



\subsection*{CPC Hardware Performance Counters}
Sun Solaris features the CPC performance counter library to query the hw-counter. See \texttt{vtcpcavail} for your options.

\subsection*{NEC SX Hardware Performance Counters}
NEC SX machines uses special register calls to query the processor's  hardware counters. See NEC SX manual.
\begin{tabular}{@{}ll@{}}
\texttt{SX\_CTR\_STM}   & System timer reg\\
\texttt{SX\_CTR\_USRCC} & User clock counter\\
\texttt{SX\_CTR\_EX}    & Execution counter\\
\texttt{SX\_CTR\_VX }   & Vector execution counter\\
\texttt{SX\_CTR\_VE}    & Vector element counter\\
\texttt{SX\_CTR\_VECC}  & Vector execution clock counter\\
\texttt{SX\_CTR\_VAREC} & Vector arithmetic execution clock counter\\
\texttt{SX\_CTR\_VLDEC} & Vector load execution clock counter\\
\texttt{SX\_CTR\_FPEC}  & Floating point data execution counter\\
\texttt{SX\_CTR\_BCCC}  & Bank conflict clock counter\\
\texttt{SX\_CTR\_ICMCC} & Instruction cache miss clock counter\\
\texttt{SX\_CTR\_OCMCC} & Operand cache miss clock counter\\
\texttt{SX\_CTR\_IPHCC} & Instruction pipeline hold clock counter\\
\texttt{SX\_CTR\_MNCCC} & Memory network conflict clock counter\\
\texttt{SX\_CTR\_SRACC} & Shared resource access clock counter\\
\texttt{SX\_CTR\_BREC}  & Branch execution counter\\
\texttt{SX\_CTR\_BPFC}  & Branch prediction failure counter
\end{tabular}

\section{Resource Usage Counters}
The Unix system call \texttt{getrusage} provides information about consumed  resources and operating system events.\\
\verb|VT_RUSAGE=ru_stime:ru_majflt|\\
The resource usage counters are not recorded at every event. Set intervall in ms:\\
\texttt{export VT\_RUSAGE\_INTV=100} (most OS support min. of 10ms )\\
Counters are per process, not per thread.
\begin{tabular}{@{}l@{}l@{ }c@{ }l@{}}
\textbf{Name}          &  \textbf{Unit}  & \textbf{Linux} & \textbf{Description}  \\
\texttt{ru\_utime}     &  ms             & x         & Total amount of user time used.  \\
\texttt{ru\_stime}     &  ms             & x         & Total amount of system time used.  \\
\texttt{ru\_maxrss}    &  kB             &           & Maximum resident set size. \\
\texttt{ru\_ixrss}     &  kB/s           &           & Integral shared memory size (text segment). \\
\texttt{ru\_idrss}     &  kB/s           &           & Integral data segment memory used over runtime. \\
\texttt{ru\_isrss}     &  kB/s           &           & Integral stack memory used over the runtime. \\
\texttt{ru\_minflt}    &  \#             & x         & Number of soft page faults. \\
\texttt{ru\_majflt}    &  \#             & x         & Number of hard page faults. \\
\texttt{ru\_nswap}     &  \#             &           & \# times process was swapped out of phys. mem. \\
\texttt{ru\_inblock}   &  \#             &           & Number of input operations via the file system. \\
\texttt{ru\_oublock}   &  \#             &           & Number of output operations via the file system. \\
\texttt{ru\_msgsnd}    &  \#             &           & Number of IPC messages sent. \\
\texttt{ru\_msgrcv}    &  \#             &           & Number of IPC messages received. \\
\texttt{ru\_nsignals}  &  \#             &           & Number of signals delivered. \\
\texttt{ru\_nvcsw}     &  \#             & x         & Number of voluntary context switches. \\
\texttt{ru\_nivcsw}    &  \#             & x         & Number of involuntary context switches. \\
\end{tabular}

\section{Function Filtering}
To decrease trace-size specify filter directives before running.\\
\verb!# syntax: <regions> -- <limit>!\\
\verb!#   regions  semicolon-separated list of regions!\\
\verb!#            (can be wildcards)!\\
\verb!#   limit    assigned call limit!\\
\verb!#             0 = region(s) denied!\\
\verb!#            -1 = unlimited!\\
\verb!add;sub;mul;div -- 1000!\\
\verb!* -- 3000000!\\
To activate set environment vaiable accordingly:\\
\texttt{export VT\_FILTER\_SPEC=/path/to/filter.txt}\\
Rank specific filtering is also possible:\\
\verb!@ 4 - 10, 20 - 29, 34!\\
\verb!foo;bar -- 2000!\\
\verb!* -- 0!\\

\section{Function Grouping}
All functions of a group are assigned one color in visualization.
\begin{tabular}{@{}ll@{}}
\textbf{Group name} & \textbf{Contained functions/regions}\\
\texttt{MPI}  & MPI functions\\
\texttt{OMP}  & OpenMP API function calls\\
\texttt{OMP\_SYNC} & OpenMP barriers \\
\texttt{OMP\_PREG} & OpenMP parallel regions \\
\texttt{Pthreads} & Pthread API function calls \\
\texttt{MEM}  & Memory allocation functions \\
\texttt{I/O}  & I/O functions \\
\texttt{LIBC} & LIBC fork/system/exec functions )\\
\texttt{Application} & all the rest\\
\end{tabular}\\
Define you own groups in a text-file.\\
\verb!# syntax: <group>=<regions>!\\
\verb!#   group    group name !\\
\verb!#   regions  semicolon-separated list of regions!\\
\verb!#            (can be wildcards)!\\
\verb!CALC=add;sub;mul;div!\\
\verb!USER=app_*!\\
To activate set environment vaiable accordingly:\\
\texttt{export VT\_GROUPS\_SPEC=/path/to/groups.txt}\\





\end{multicols}
\end{document}

